%------------------------------------------------------------------------------
%
%	abstract.tex General theory part
%
%	INCLUDE FILE FOR LaTeX2e DOCUMENT
%
%	AUTHOR: Ari Potkonen /JARVENPAA/ Sat Dec 9 2023
%------------------------------------------------------------------------------
%         1         2         3         4         5         6         7
%123456789012345678901234567890123456789012345678901234567890123456789012345678
%-BEGIN OF INCLUDE FILE--------------------------------------------------------
%\part{Abstract}
%\begin{abstract}
%\pagestyle{plain}
%\pagenumbering{gobble}
\addcontentsline{toc}{chapter}{Abstract}
General Theory Back Cover Abstract\hfill
\label{abstract}
\index{abstract}

%\linebreak

Root question here is that do we understand spacetime curvature around energy
concentration when energy is small and dimensions are small on subatomic and
atomic scale.

This booklet breaks traditional method based description from the physics
around as. Aim is have discussion, drop statistical methods for while for now
and go back to the philosophy basics. Basic idea is well known from
mathematics: Guess the correct answer and try to proof it to be true. In this
booklet we really only lift the idea up, because others have done years and
years work around these issues and we do not have solution yet.

Mostly what's done here is some proposal from reclassification of terms/items.
Existing statistical method based theory verified by experiments, is not
anyway wrong, but it's wide use prohibits, to ask meaningful questions. Or yes
those have been asked, but not really seriously researched. Now I again lift
these obvious questions and try to propose guessed, but obvious solution for
accurate mathematical formulation. We need mathematical model capable to
explain time dependent behavior on local coordinate system and how it
transforms to lightspeed frame in subatomic, atomic scale. not just the
statistical representation of it as now with existing models.

If exact mathematical solution is not found, we should have at least enough to
model and find computational solution for the simplest examples we could
think.
 
Initially we do initial assumption that all what we have is spacetime,
electromagnetism and energy. Nothing else is there, even we have told to
ourselves something else. Then we start to go through details case by case
by thinking does it be possible and what it means in this particular case.
%like 0.511 MeV - 938.272 MeV - 939.565 MeV
%\linebreak \vfill
%\includegraphics[height=32pt,width=32pt]{fig/.eps} \hfill
%\includegraphics[height=32pt,width=32pt]{fig/.eps} \hfill
%\includegraphics[height=32pt,width=32pt]{fig/.eps} \hfill
%\includegraphics[height=32pt,width=32pt]{fig/.eps} \hfill
%\includegraphics[height=32pt,width=32pt]{fig/.eps} \hfill
%\includegraphics[height=32pt,width=32pt]{fig/.eps}

%\end{abstract}
%-END OF INCLUDE FILE----------------------------------------------------------
