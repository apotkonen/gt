%-------------------------------------------------------------------------------
%
%	formulat.tex Document formulation part
%
%	INCLUDE FILE FOR LaTeX2e DOCUMENT
%
%	AUTHOR: Ari Potkonen /JARVENPAA/ Mon Dec 11 2023
%------------------------------------------------------------------------------
%         1         2         3         4         5         6         7
%123456789012345678901234567890123456789012345678901234567890123456789012345678
%-BEGIN OF INCLUDE FILE--------------------------------------------------------
\begin{comment}\end{comment}
\chapter{Mathematical formulation}

The most needed gap fillings for Nuclides chart\cite{Nuclides} are now done,
and it's time to be able to understand fully local space of electron and
positron. How space is curled and how electromagnetic $\gamma$ -ray lives
there according the Maxwell's equations in curved
spacetime\cite{MaxwellsEquations}.

Dynamic process of electron-positron annihilation\cite{Annihilation}
opening the curls and releasing of $\gamma$ -rays. Pair
production\cite{PairProduction} starting from "triplet production".

Continuing with the proton, antiproton, neutron, antineutron static
representations without and with superpositioned electrons or positrons.

Then proceeding what happens on other collected Feynmann
diagrams\cite{FeynmannDiagram}.

You are welcomed to continue this work and fill the missing.

\section{Static electron and static positron}
\label{electron_positron}
\index{electron}
\index{positron}

There is need to understand electron and positron local space curling,
electric field in curled space and how this looks from remote static
observer viewpoint. All in detail.

\section{Annihilation}
\label{annihilation}
\index{annihilation}

Original curls in detail, how waves summing at cross section of curls
and how this leads to curl opening and $\gamma$ -rays release.

Curved space "Gravity" behaviour during curl curvature opening.

\section{Pair production}
\label{pair_production}
\index{pair production}

Incoming $\gamma$ -ray wavelets. Curl creation in electron collision. Curl
own space curvature development and from observers point static situation
after that.

Formulation of discrete curl curvature form, and it's relation to gravity.

\begin{comment}\end{comment}
%-END OF INCLUDE FILE----------------------------------------------------------
